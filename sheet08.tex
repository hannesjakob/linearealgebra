\documentclass[14pt]{Article}
\usepackage[a4paper, total={6in,8in}]{geometry}
\usepackage{amssymb}
\usepackage{amsmath}
\usepackage{fancyhdr}
\usepackage{titlesec}
\titleformat{\subsubsection}[runin]{}{}{}{}[]
\pagestyle{fancy}
\fancyhead[L]{
	Nils Lauinger
	}
\fancyhead[C]{
	Analysis 1 {\"U}bung 8}
\fancyhead[R]{Florian Kramer \\ Hannes Jakob}
\geometry{legalpaper,margin=1in}
\linespread{1.5}
\begin{document}
\subsection*{Aufgabe 33}
Sei \(B = ((v_i, 0))_{i \in I} \cup (w_j, 0))_{j \in J}\). Seien o.B.d.A. I und J disjunkt (sind sie das nicht kann man durch umbenennen der Elemente und entsprechendem Anpassen der Abbildung auf \(w_j\) die Schnittmenge leeren, ohne dabei \(w_j\) zu \"andern). Dann ist zu zeigen:
\begin{enumerate}
\item{B ist ein Erzeugendensystem von \((V \times W)\)\\
Sei \(x = (a, b) \in V \times W\). Dann existieren \((\lambda_i)_{i\in I}, (\mu_j)_{j \in J}\), so dass\\
\(a = \sum_{i \in I} \lambda_i v_i, b = \sum_{j \in J}\mu_j w_j\).\\
Damit gilt aber\\
\((a, b) = (\sum_{i \in I} \lambda_i v_i, \sum_{j \in J}\mu_j w_j) = \sum_{i \in I \cup J} \xi_i b_i\)\\
mit \(\xi_i = \begin{cases}\lambda_i & i \in I\\\mu_i&i \in J\end{cases}, b_i = \begin{cases}(v_i, 0) & i \in I\\(0, w_i)&i \in J\end{cases} \in B\).\\
Also l\"asst sich jedes \(x \in V \times W\) als Linearkombination von Elementen aus B darstellen, und B ist ein Erzeugendensystem von \(V \times W\).
}
\item{B ist linear unabh\"angig.\\
Seien\\
\(\sum_{i \in I \cup J} \xi_i b_i = \sum_{i \in I} \lambda (v_i, 0) + \sum_{j \in J} \mu_j (0, w_j) = (0, 0)\), \(\xi\) und b wie oben.\\
Dann ist sofort ersichtlich, dass beide Summen \((0, 0)\) sein m\"ussen, damit das Ergebnis \((0, 0)\) ist. Da alle \((v_i)_{i \in I}\) und \((w_j)_{j \in J}\) jedoch linear unabh\"angig sind, folgt direkt, dass \(\lambda_i = 0, \mu_j = 0\), also auch \(\xi_i = 0\), womit B linear unabh\"angig ist.
}
\end{enumerate}
Damit ist B eine Basis von \(V \times W\). Da \(((v_i, 0))_{i \in I}, ((0, w_j))_{j \in J}\) nach Konstruktion Disjunkt sind, gile \(|B| = dim(V \times W) = dim(V) + dim(W)\), falls V und W endlich dimensional sind.
\subsection*{Aufgabe 34}
$\lambda_1 * (a+b+c)+\lambda_2 * (2a+2b+2c-d)+ \lambda_3 * (a-b-e)+\lambda_4*(5a+6b-c+d+e)+\lambda_5*(a-c+3e)+\lambda_6 *(a+b+d+e)=0\\
\begin{pmatrix} 
1 & 2 & 1 & 5 & 1 & 1\\
1 & 2 & -1 & 6 & 0 & 1 \\
1 & 2 & 0 & -1 & -1 & 0\\
0 & -1 & 0 & 1 & 0 & 1 \\
0 & 0 & -1 & 1 & 3 & 1
\end{pmatrix} * 
\begin{pmatrix}
\lambda_1\\\lambda_2\\\lambda_3\\\lambda_4\\\lambda_5\\\lambda_6
\end{pmatrix} =
\begin{pmatrix}
0\\0\\0\\0\\0
\end{pmatrix} = 
\begin{pmatrix} 
1 & 2 & 1 & 5 & 1 & 1\\
1 & 2 & -1 & 6 & 0 & 1 \\
1 & 2 & 0 & -1 & -1 & 0\\
0 & -1 & 0 & 1 & 0 & 1 \\
0 & 0 & -1 & 1 & 3 & 1
\end{pmatrix}
\begin{pmatrix}
8\\\frac{-10}{3}\\\frac{-1}{3}\\\frac{1}{3}\\1\\\frac{-11}{3}
\end{pmatrix}=
\begin{pmatrix}
8-\frac{20}{3}-\frac{1}{3}+\frac{5}{3}+1-\frac{11}{3}\\
8-\frac{20}{3}+\frac{1}{3}+\frac{6}{3}-\frac{11}{3}\\
8-\frac{20}{3}-\frac{1}{3}-1\\
\frac{10}{3}+\frac{1}{3}-\frac{11}{3}\\
\frac{1}{3}+\frac{1}{3}+3-\frac{11}{3}
\end{pmatrix}
\\
$Im Folgenden meint das Zeichen $\sim$, dass die folgende Matrix eine Umformung der Vorherigen ist\\$
\begin{pmatrix} 
1 & 2 & 1 & 5 & 1 & 1\\
1 & 2 & -1 & 6 & 0 & 1 \\
1 & 2 & 0 & -1 & -1 & 0\\
0 & -1 & 0 & 1 & 0 & 1 \\
0 & 0 & -1 & 1 & 3 & 1
\end{pmatrix} \sim
\begin{pmatrix}
1 & 2 & 1 & 5 & 1 & 1\\
0 & 0 & -2 & 1 & -1 & 0\\
0 & 0 & -1 & -6 & -2 & -1\\
0 & -1 & 0 & 1 & 0 & 1\\
0 & 0 & -1 & 1 & 3 & 1
\end{pmatrix} \sim 
\begin{pmatrix}
1 & 2 & 1 & 5 & 1 & 1\\
0 & 1 & 0 & -1 & 0 & -1\\
0 & 0 & -1 & -6 & -2 & -1\\
0 & 0 & -2 & 1 & -1 & 0\\
0 & 0 & -1 & 1 & 3 & 1
\end{pmatrix} \\\sim
\begin{pmatrix}
1 & 2 & 1 & 5 & 1 & 1\\
0 & 1 & 0 & -1 & 0 & -1\\
0 & 0 & 1 & 6 & 2 & 1\\
0 & 0 & -2 & 1 & -1 & 0\\
0 & 0 & -1 & 1 & 3 & 1
\end{pmatrix} \sim
\begin{pmatrix}
1 & 2 & 1 & 5 & 1 & 1\\
0 & 1 & 0 & -1 & 0 & -1\\
0 & 0 & 1 & 6 & 2 & 1\\
0 & 0 & 0 & 13 & 3 & 2\\
0 & 0 & 0 & 7 & 5 & 2
\end{pmatrix} \sim 
\begin{pmatrix}
1 & 2 & 1 & 5 & 1 & 1\\
0 & 1 & 0 & -1 & 0 & -1\\
0 & 0 & 1 & 6 & 2 & 1\\
0 & 0 & 0 & 7 & 5 & 2\\
0 & 0 & 0 & 13 & 3 & 2
\end{pmatrix} \\\sim
\begin{pmatrix}
1 & 2 & 1 & 5 & 1 & 1\\
0 & 1 & 0 & -1 & 0 & -1\\
0 & 0 & 1 & 6 & 2 & 1\\
0 & 0 & 0 & 1 & 5/7 & 2/7\\
0 & 0 & 0 & 13 & 3 & 2
\end{pmatrix} \sim
\begin{pmatrix}
1 & 2 & 1 & 5 & 1 & 1\\
0 & 1 & 0 & -1 & 0 & -1\\
0 & 0 & 1 & 6 & 2 & 1\\
0 & 0 & 0 & 1 & 5/7 & 2/7\\
0 & 0 & 0 & 0 & -44/7 & 12/7
\end{pmatrix}$\\
W{\"a}hle $\lambda_5 = 1 \Rightarrow \lambda_6 = -\frac{11}{3}\\ \Rightarrow \lambda_4=-\frac{5}{7}+\frac{11}{3}*\frac{2}{7}=\frac{1}{3}\\
\Rightarrow \lambda_3=-6\frac{1}{3} -2(1)-1\frac{11}{3}= -\frac{1}{3}\\ \Rightarrow \lambda_2=\frac{1}{3}+\frac{11}{3}=4\\ \Rightarrow\lambda_1=4+\frac{1}{3}-5\frac{1}{3}-1-\frac{11}{3}=3+\frac{15}{3}=8
$
\subsection*{Aufgabe 35}
\subsubsection*{\textbf{Satz 1:}}$\forall x \in M: 0*x=0, \forall\lambda{\in}R: \lambda * 0=0$\\
$0*x=(0+0)*x=0*x+0*x{\Rightarrow}0*x=0$
\subsubsection*{\textbf{Satz 2:}}$\forall x \in M: (-1)*x=-x$\\
$0=0*x=(1-1)*x=1*x+(-1)*x=x+(-1)*x=0\Rightarrow(-1)*x=-x$
\subsubsection*{\textbf{Satz 3:}} Zeigen sie, dass im Allgemeinen nicht für Links-Moduln gilt: $\lambda{\in}R, x{\in}M, \lambda*x=0\Rightarrow\lambda=0{\vee}x=0$
Gegenbeispiel: Sei $R=4\mathbb{Z}$ und $M=R^1 \Rightarrow{\exists}2{\in}R, 2{\in}M: \bar{2}*\bar{2}=\bar{4}=0$
\end{document}
