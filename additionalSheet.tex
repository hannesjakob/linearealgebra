\documentclass[12pt]{article}
\usepackage[a4paper, total={6in,8in}]{geometry}
\usepackage{amssymb}
\usepackage{amsmath}
\usepackage{fancyhdr}
\usepackage{titlesec}
\titleformat{\subsubsection}[runin]{\bfseries}{}{}{}[]
\titleformat{\subsection}[runin]{\bfseries}{}{}{}[]
\pagestyle{fancy}
\fancyhead[L]{
	Nils Lauinger
	}
\fancyhead[C]{
	Analysis 1 Extrablatt}
\fancyhead[R]{Florian Kramer \\ Hannes Jakob}
\geometry{legalpaper,margin=1in}
\linespread{1.5}
\begin{document}
\subsection*{Aufgabe 44}\textit{Polynome}\\
Sei $V$ der Raum der Polynome vom Grad h{\"o}chstens 3 auf $\mathbb{R}$ und $\mathfrak{B}$ = $\{b_i\}_{i=1,...,4}=\{1,t,t^2,t^3\}$. Wir betrachten die linearen Abbildungen
$$F: V\rightarrow\mathbb{R}: f \rightarrow \int_{-1}^1f(t)dt \texttt{ und } G: V \rightarrow \mathbb{R}, f\rightarrow(f(-1),f(0),f(1)).$$
\subsubsection*{(i)}F{\"u}r $i=1,2,3,4$ bestimmen sie die Entwicklungskoeffizienten von $F(b_i)$ bzgl. dem kanonischen Basisvektor $(e_1)$ von $\mathbb{R}$ und von $G(b_i)$ bzgl. der kanonischen Basisvektoren $\{e_1, e_2, e_3\}$ (von $\mathbb{R}^3$)
\subsubsection*{(ii)}Zeigen sie: Ker$G \subset$ Ker $F$
$\forall x \in Ker(G): G(x)=(0,0,0) \Rightarrow f(-1)=0, f(0)=0, f(1)=0, f=\lambda_1t3+\lambda_2t^2+\lambda_3t+\lambda_4, f(0)=0 \Rightarrow \lambda_4=0, f(-1)=-\lambda_1+\lambda_2-\lambda_3=0 \Rightarrow \lambda_1=\lambda_2-\lambda_3, f(1)=\lambda_1+\lambda_2+\lambda_3=2*\lambda_2=0 \Rightarrow \lambda_2=0 \Rightarrow \lambda_1=-\lambda_3 \Rightarrow f(x)=\lambda_1x^3-\lambda_1x \Rightarrow \int_{-1}^1f(x)=(\frac{\lambda_1}{4}x^4-\frac{\lambda_1}{2}x^2)_{-1}^1=(\frac{\lambda_1}{4}-\frac{\lambda_1}{2})-(\frac{\lambda_1}{4}-\frac{\lambda_1}{2})=0 \Rightarrow \forall x\in V, G(x)=0: F(x)=0 \Rightarrow Ker(G)\subset Ker(F)$
\subsubsection*{(iii)}Finden sie eine Abbildung $H: \mathbb{R}^3 \rightarrow \mathbb{R}$ mit $H \circ G =F$\\
$f:= \lambda_1*t^3+\lambda_2*t^2+\lambda_3*t+\lambda_4G: (f(-1),f(0),f(1) \rightarrow \int_{-1}^1=F(1)-F(-1)=(\lambda_1+\lambda_2+\lambda_3+\lambda_4)-(\lambda_1-\lambda_2+\lambda_3-\lambda_4)=2*(\lambda_2+\lambda_4)$\\
$f(0)=\lambda_4, f(-1)=-\lambda_1+\lambda_2-\lambda_3+\lambda_4 \Rightarrow \lambda_1=-f(-1)+\lambda_2-\lambda_3+f(0)$\\$f(1)=\lambda_1+\lambda_2+\lambda_3+\lambda_4=(-f(-1)+\lambda_2-\lambda_3+f(0))+\lambda_2+\lambda_3+f(0)\\=-f(-1)+2*\lambda_2+2*f(0) \Rightarrow \lambda_2 = \frac{f(1)-2f(0)+f(-1)}{2} \Rightarrow H(f(-1),f(0),f(1))=f(1)-2f(0)+f(-1)+2*f(0)=f(1)+f(-1)$\\
Test: $f:=x^3+x^2+x+1, G(f)=(f(-1),f(0),f(1))=(0,1,4) \Rightarrow H((0,0,4))=4+0-0=4, F(f)=\int_{-1}^1f(t)dt=[\frac{x^4}{4}+\frac{x^3}{3}+\frac{x^2}{2}+x]_{-1}^1=(\frac{1}{4}+\frac{1}{3}+\frac{1}{2}+1)-(\frac{1}{4}-\frac{1}{3}+\frac{1}{2}-1)=\frac{8}{3}$
\end{document}
