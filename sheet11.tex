\documentclass[12pt]{article}
\usepackage[a4paper, total={6in,8in}]{geometry}
\usepackage{amssymb}
\usepackage{amsmath}
\usepackage{fancyhdr}
\usepackage{titlesec}
\titleformat{\subsubsection}[runin]{\bfseries}{}{}{}[]
\titleformat{\subsection}[runin]{\bfseries}{}{}{}[]
\pagestyle{fancy}
\fancyhead[L]{
	Nils Lauinger
	}
\fancyhead[C]{
	Analysis 1 {\"U}bung 8}
\fancyhead[R]{Florian Kramer \\ Hannes Jakob}
\geometry{legalpaper,margin=1in}
\linespread{1.5}
\begin{document}
\subsection*{Aufgabe 48} \textit{Isomorphies{\"a}tze}\\
Es seien $V$ ein $K$-Vektorraum und $V_1, V_2 {\in}V$ Untervektorr{\"a}ume. Zeigen sie, dass gilt:
\subsubsection*{(i)}$$(V_1+V_2)/{V_1}{\cong}V_2/(V_1{\cap}V_2)$$
Es sei $F: V_2 \rightarrow (V_1+V_2)/V_1, x\rightarrow[x]$. Dann gilt:
$\forall x: (x \in V_2): F(x)=0 \Rightarrow x-0 \in V_1 \Rightarrow x \in V_1 \Rightarrow x \in (V_2 \cap V_1) \Rightarrow ker(F)=(V_2 \cap V_1)$\\
$\forall x \in (V_1+V_2): x \in span(V_1 \cup V_2) \Rightarrow \exists v_1 \in span(V_1)=V_1, v_2 \in span(V_2)=V_2: x=v_1+v_2 \Rightarrow [x]=[v_1+v_2]=[v_1]+[v_2]=[v_2]$(da $v_1 \in V_1 \Rightarrow [v_1]=0) \Rightarrow (V_1+V_2)/V_1=V_2/V_1$\\
Daraus folgt trivialerweise, dass $F(V_2)=Im(F)=V_2/V_1=(V_1+V_2)/V_1$\\
Nach dem Homomorphiesatz existiert dann ein Isomorphismus$$\bar{F}: V_2/ker(F)=V_2/(V_1\cap V_2)\rightarrow(V_1+V_2)/V_1=Im(F),$$ womit $V_2/(V_1\cap V_2)\cong(V_1+V_2)/V_1$ gilt.
\subsubsection*{(ii)}Falls $V_1{\subset}V_2{\subset}V:$$$(V/V_1)/(V_1/V_2){\cong}V/V_2$$
Es sei $F:V/V_1 \rightarrow V/V_2: [x]\rightarrow[[x]]$. Dann gilt:
$\forall [x] {\in}(V_2/V_1): [x]-[0]=[x-0]=[x] \in V_2 \Rightarrow F([x])=0 \Rightarrow ker(F)=(V_2/V_1)$\\
$\forall[x]\in (V/V_1):  x-[x] \in V_1 \Rightarrow x-[x] \in V_2 \Rightarrow F([x])=[[x]] \in (V/V_2) \Rightarrow F((V/V_1))=Im(F)=(V/V_2)$\\
Nach dem Homomorphiesatz existiert dann ein Isomorphismus $$\bar{F}: (V/V_1)/ker(F)=(V/V_1)/(V_1/V_2)\rightarrow(V/V_2)=Im(F),$$ womit $(V/V_1)/(V_1/V_2)\cong V/V_2$ gilt.
\subsection*{Aufgabe 49}\textit{Matrixmultiplikation}\\
Bilden sie alle m{\"o}glichen Matrixprodukte (auch Dreifachmultiplikationen) aus den folgenden 3 Matrizen und berechnen sie diese.\\
$$A=\begin{pmatrix}
-1 & 3 & 100 & -2 & 5 \\
0 & -3 & 1 & 7 & 2\\
3 & 1 & 2 & 0 & 6
\end{pmatrix} \in \mathbb{R}^{3\times5}, 
B=\begin{pmatrix}
98 & 1 \\ -5 & 7 \\ 12 & 5 \\ 0 & 4
\end{pmatrix} \in \mathbb{R}^{4\times2},
C=\begin{pmatrix}
-2 & 1 & 3 \\ -4 & 2 & 6
\end{pmatrix} \in \mathbb{R}^{2\times3}$$
$C*A=\begin{pmatrix}
11 & -6 & -193 & 11 & 10 \\
22 & -12 & -386 & 22 & 20
\end{pmatrix} \in \mathbb{R}^{2\times5},
B*C*A=\begin{pmatrix}
1100 & -600 & -19300 & 1100 & 1000 \\
99 & -54 & 1737 & 99 & 90 \\
242 & -132 & 4246 & 242 & 220 \\
88 & -48 & -1464 & 88 & 80
\end{pmatrix}\\
B*C=\begin{pmatrix}
-200 & 100 & 300 \\
-18 & 9 & 27 \\
-44 & 22 & 66 \\
-16 & 8 & 24
\end{pmatrix} \in \mathbb{R}^{4\times3}$\\
Andere Produkte sind aufgrund der Dimensionen der Matrizen nicht m{\"o}glich.
\subsection*{Aufgabe 50}\textit{Unterringe}\\
Sind die folgenden Teilmengen Unterringe:\\
Bemerkung: Es muss nur die Multiplikation {\"u}berpr{\"u}ft werden, da aus der Eintragsweisen Addition folgt, dass die jeweiligen vermeintlichen Unterringe zumindest Untergruppen sind.
\subsubsection*{(1)}$M_1:=\{a_{ij}{\in}Mat_K(n{\times}n): a_{ij} = 0, i{\geq}j\}\subset Mat_K(n{\times}n)$\\
$A{\in}M_1, B{\in}M_1, A*B=C=(c_{ij}), c_{ij}=\sum_{k=1}^na_{ik}b_{kj}=\sum_{k=1}^{(i-1)}a_{ik}b_{kj}+\sum_{k=i}^na_{ik}b_{kj}$\\
Sei $i{\geq}j: \sum_{k=1}^{(i-1)}a_{ik}b_{kj}=\sum_{k=1}^{(i-1)}0*b_{kj}=0$, da $k{\leq}(i-1)<i \Rightarrow a_{ik}=0$\\
$\sum_{k=i}^{n}a_{ik}b_{kj}=\sum_{k=i}^{n}a_{ik}*0=0$, da $k{\geq}i{\geq}j \Rightarrow b_{kj}=0$\\
$\Rightarrow A*B \in M_1$ und $\{M_1, +, *\}$ ist ein Unterring
\subsubsection*{(2)}$M_2:=\{a_{ij}{\in}Mat_K(n{\times}n): a_{ij}=0, i{\geq}j+k, j{\geq}i+k, k\in\mathbb{N} \subset Mat_K(n{\times}n)$\\
$A=B=\begin{pmatrix}
1 & 1 & 0 \\ 1 & 1 & 1 \\ 0 & 1 & 1
\end{pmatrix} \in M_2$ (f{\"u}r $k=2$,
$A*B=\begin{pmatrix}
2 & 2 & 1 \\ 2 & 3 & 2 \\ 1 & 2 & 2
\end{pmatrix}{\notin}M_2$\\
$\Rightarrow M_2$ ist kein Unterring.
\subsection*{(3)}$M_3:=\{\begin{pmatrix}0 & a \\ 0 & b 	\end{pmatrix}: a,b \in K, \}\subset Mat_K(2\times2)$\\
$A=\begin{pmatrix}0 & a_1 \\ 0 & b_2\end{pmatrix}{\in}M_3, B=\begin{pmatrix}0 & a_2 \\ 0 & b_2\end{pmatrix}{\in}M_3 \Rightarrow A*B= \begin{pmatrix}0 & a_1*b_2 \\ 0 & b_1*b_2\end{pmatrix}{\in}M_3\\
\Rightarrow M_3$ ist ein Unterring.
\end{document}
